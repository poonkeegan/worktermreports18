\documentclass[12pt, letterpaper]{article}
\usepackage[utf8]{inputenc}
\usepackage[hidelinks]{hyperref}
\usepackage{indentfirst}

\setcounter{section}{-1}
\hypersetup{linktoc=all}

\title{Advantages and Disadvantages of Android}
\author{Keegan Poon, Third Year}
\date{May - August 2018, First Workterm}
 
\begin{document}
 
\begin{titlepage}
\maketitle
\end{titlepage}
 
\section{Summary}
This report will focus on analyzing the various positives and negatives
that occur when developing software for Android when considered in the
setting of this work term. Topics include the variety and ubiquity of 
android devices available in the market, as well as code development 
environment that is prevalent for writing Android software. These are 
prominent points to consider when developing software, so their analysis
will prove useful to consider.

\newpage

\tableofcontents
\newpage

\section{Introduction}

The objectives for this report, are to discuss some of the more prominent
things about Android development that are likely to influence opinions.
Some of these include the incredible userbase of the platform, its
diversity of utilities and the diversity of the products themselves.
They are chosen as they are all topics that continual development on the
Android platform will cause to manifest, likely being behind many of
the reasons and decisions made for development.

\section{Work Term Overview}

The work term was carried out at BBMtek which develops and maintains the 
BBM Android and iPhone application, as well as develop the beta desktop 
version. The work term position was Android Software Developer (co-op) 
which consisted of working on new features, investigating issues that occur
and fixing bugs that come up on the Android version of the application.

The company consists of multiple development teams that each work on
their own cordoned off portions of the app, though there is frequent
collaborations between teams when necessary. For the team which the
work term took place in, the vast majority of responsibilities lied
within the realm of messaging connectivity, so to do with creating and
maintaining a connection between the application and the servers responsible
for broadcasting messages to devices. On a daily basis, the general work
responsibilities are to develop features and fixes for the next release, as
well as look into issues that are raised by the support team so that
causes and potential fixes to problems can be identified and applied.

There are multiple responsibilities that the position covers, all relating
to software development. An example is changing the user interface of the 
notifications to have more actions, allowing you to add contacts or post 
without having to navigate through the app menus. Another would be 
conducting investigations into the application as to why certain people 
are not receiving messages, and proceeding to fix the underlying issue.
On a weekly or biweekly basis, these responsibilities are cycled through
as the Agile methodology progresses from stage to stage each sprint.
This is to ensure that work done by developers is relevant, correct, and on
track to be completed in a short time frame.

Overall, the company is a fairly standard software development environment
with Agile standups every day, and progress meetings every week. The only 
real specifics are that the work term role is tied to Android and hence, 
Java and Kotlin, as well as the fact that the team's role is more focused 
around interaction between the client and server, not so much on the client
interface and experience or any other part in particular.

\section{Report}

\subsection{Presence in the market}
This is one of the biggest reasons overall for picking a platform to
develop for. Nowadays Aside from Apple phones, virtually any other
smartphone OSes are nearly unheard-of. Further more, the market share of
Android is so big as to eclipse Apple\cite{marketshare}. From that, it
is evident that Android gives a wide user base across the entire world
that can be marketed towards. In a sense, not developing for Android can
be business suicide for mobile application development unless a strong
monetization model is implemented ensuring a large amount of earnings.

Not only is Android popular in general, but it is the main choice in
lower end phones. Since the market for BBM largely consists of older
phones rather than newer more expensive devices, it again means that
the lion's share of the consumers would rather an Android app to 
any other choices.



\subsection{Ease of development}
As well as Android being an extremely standardized operating system, the
tools to develop with it are also very common. The main language in use
for development is Java, which to this day is used all over the place
\cite{java} within Github, companies maintaining legacy systems, and even
taught in UTSC core programming courses. This makes it easier to find and
keep developers that are proficient with Android due to the large presence
of people that have learned it, or are learning it due to the current large
demand.

The fact that the languages are so common and popular, also means that the
hardware required to develop the product are available and common place.
Virtually every modern computer is able to have a development environment
for Android setup and running, meanwhile developing for a platform such as
iOS requires the company purchase a large amount of Macintosh computers
so that the developers are able to develop, since Apple keeps their
proprietary tools to themselves and their walled garden ecosystem.

The software setup for development is also fairly straightforward for
Android, with freely available integrated development environments such
as Android Studio being constantly developed. This in turn feeds into being
the top development platforms, because it is so easy to get up and running.
Therefore, when the developers do join the team, everything from the
language, to the platform they will develop for, to the software being used
to develop for it will all be completely familiar to them, and much less
training will be required to have them integrate into the projects.

Furthermore, since Java is supported many other JVM languages which easily
interoperate with Java are also easily accessable, allowing for much greater
flexibility than needing potentially less reliable support for allowing
other languages of choice that match the problem domain.

\subsection{Fragmentation of the hardware ecosystem}
Not everything on Android is amazingly spectacular. The fact that nearly
all the phones today run Android also has some drawbacks, namely that
every type of phone has some type of Android, but the exact matches are
never standardized.

In other words, across all the Android users, they all have a large variety
in their versions of Android, some having versions that are years old,
some having the latest update, along with hundreds of different types of
hardware running the versions of Android, with their own particular quirks
and issues specific to them.

Not only does this cause trouble for simply fixing defects that occur in
certain versions of applications, but when developing new features all the
old versions must also be held in account. When an API level, or specific
software version is chosen to maintain support for, all new features must
restrict themselves to simply the feature set found in these versions
otherwise locking out the previous userbase from being able to continue
using the app.

That isn't to say there isn't real problems with the hardware and specific
brands that make them up. For example, Xiaomi and One Plus are notorious
for having hard to work with phones due to their incredibly restrictive
specialized operating systems. They enforce much stricter app usage policies
then the default Android operating system, and even then may implement the
default features in different ways, causing issues to any developers who
do not have the resources to test on the wide breadth of phones Android
phones that exist.

\section{Conclusion}

\section{Appendices}

\begin{thebibliography}{9}
\bibitem{marketshare}
Don Reisinger: Apple Trails Samsung in Smartphone Market - And Won't Catch Up in 2018
\\\texttt{http://fortune.com/2018/02/13/apple-iphone-samsung-market-share/}
\bibitem{java}
Ben Putano: Most Popular and Influential Programming Languages of 2018
\\\texttt{https://stackify.com/popular-programming-languages-2018/}
\end{thebibliography}
\end{document}

