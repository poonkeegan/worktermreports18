\documentclass[12pt, letterpaper]{article}
\usepackage[utf8]{inputenc}
\usepackage[hidelinks]{hyperref}
\usepackage{indentfirst}

\setcounter{section}{-1}
\hypersetup{linktoc=all}

\title{Advantages and Disadvantages of Android}
\author{Keegan Poon, Third Year}
\date{May - August 2018, First Workterm}
 
\begin{document}
 
\begin{titlepage}
\maketitle
\end{titlepage}
 
\section{Summary}
This report will focus on analyzing the various positives and negatives
that occur when developing software for Android when considered in the
setting of this work term. Topics include the variety of android devices
available in the market, as well as code development enviroment that is
prevalent for writing Android software. 



\newpage

\tableofcontents
\newpage

\section{Introduction}

\section{Work Term Overview}

The work term was carried out at BBMtek which develops and maintains the 
BBM Android and iPhone application, as well as develop the beta desktop 
version. The work term position was Android Software Developer (co-op) 
which consisted of working on new features, investigating issues that occur
and fixing bugs that come up on the Android version of the application.

The company consists of multiple development teams that each work on
their own cordoned off portions of the app, though there is frequent
collaborations between teams when necessary. For the team which the
work term took place in, the vast majority of responsibilities lied
within the realm of messaging connectivity, so to do with creating and
mainting a connection between the application and the servers responsible
for broadcasting messages to devices. On a daily basis, the general work
reponsibilties are to develop features and fixes for the next release, as
well as look into issues that are raised by the support team so that
causes and potential fixes to problems can be identified and applied.

There are multiple responsibilities that the position covers, all relating
to software development. An example is changing the user interface of the 
notifications to have more actions, allowing you to add contacts or post 
without having to navigate through the app menus. Another would be 
conducting investigations into the application as to why certain people 
are not receiving messages, and proceeding to fix the underlying issue.
On a weekly or biweekly basis, these responsibilities are cycled through
as the Agile methodology progresses from stage to stage each sprint.
This is to ensure that work done by developers is relevant, correct, and on
track to be completed in a short time frame.

Overall, the company is a fairly standard software development environment
with Agile standups every day, and progress meetings every week. The only 
real specifics are that the work term role is tied to Android and hence, 
Java and Kotlin, as well as the fact that the team's role is more focused 
around interaction between the client and server, not so much on the client
interface and experience or any other part in particular.

\section{Report}

\section{Conclusion}

\section{Appendices}



\end{document}

